%=========================================================================
% (c) Michal Bidlo, Bohuslav Křena, 2008


\chapter{Úvod}
\todo{Shrnutí cílů}
\blind[1]
\todo{Teorie, proč se tím zaobírat}
\blind[1]
\todo{Návrh a implementace}
\blind[1]
\todo{Vyhodnocení}
\blind[1]


\chapter{Teoretický rozbor}
\todo{Popis kapitoly}
\blind[1]

\section{Požadavky na návrh software}
\todo{Délka, složitost. Potřeba oddělení částí.}
\blind[2]

\section{Datově orientovaný návrh}
\todo{Proč se používá, architektura PC.}
\blind[4]

\section{Reprezentace entit a jejich chování}
\todo{Co to je entita, chování}
\blind[2]

\subsection{Objektově orientovaná hierarchie}
\todo{Popis, příklad, vyhodnocení výhod/nevýhod}
\begin{figure}[H]
	\tmpframe{\includegraphics[width=\linewidth]{TODO-image}}
	\caption{Příklad objektově orientované hierarchie. \todo{Vytvořit}}
\end{figure}
\begin{figure}[H]
	\tmpframe{\includegraphics[width=\linewidth]{TODO-image}}
	\caption{Obsah paměti u objektově orientovaných hierarchií. \todo{Vytvořit}}
\end{figure}
\blind[3]

\subsection{Objektově orientovaná kompozice}
\todo{Popis, příklad, vyhodnocení výhod/nevýhod}
\begin{figure}[H]
	\tmpframe{\includegraphics[width=\linewidth]{TODO-image}}
	\caption{Příklad objektově orientované kompozice. \todo{Vytvořit}}
\end{figure}
\begin{figure}[H]
	\tmpframe{\includegraphics[width=\linewidth]{TODO-image}}
	\caption{Obsah paměti u objektově orientované kompozice. \todo{Vytvořit}}
\end{figure}
\blind[3]

\subsection{Datově orientovaná kompozice}
\todo{Popis, příklad, vyhodnocení výhod/nevýhod}
\begin{figure}[H]
	\tmpframe{\includegraphics[width=\linewidth]{TODO-image}}
	\caption{Příklad datově orientované kompozice. \todo{Vytvořit}}
\end{figure}
\begin{figure}[H]
	\tmpframe{\includegraphics[width=\linewidth]{TODO-image}}
	\caption{Obsah paměti u datově orientované kompozice. \todo{Vytvořit}}
\end{figure}
\blind[4]

\section{Komponentní systémy}
\todo{Zaklad, co to je.}
\blind[1]

\subsection{Historie vývoje}
\blind[3]

\subsection{Koncepty}
\todo{Představení základní architektury návrhového vzoru - entity, komponenty, systémy}
\begin{figure}[H]
	\tmpframe{\includegraphics[width=\linewidth]{TODO-image}}
	\caption{Entita reprezentována jako řádek v databázi. \todo{Vytvořit}}
\end{figure}
\blind[5]

\subsection{Výhody zvolené reprezentace}
\todo{Lokalita dat, instrukcí. Cache. Dynamičnost, prototypování. Serializace. Embedded skriptovací jazyky.}
\blind[4]

\section{Paralelizmus}
\todo{Důležitost, hardware - cache, popis možností paralelizace.}
\blind[2]
\begin{figure}[H]
	\tmpframe{\includegraphics[width=\linewidth]{TODO-image}}
	\caption{Paralelizmus uvnitř systémů. \todo{Vytvořit}}
\end{figure}
\blind[1]
\begin{figure}[H]
	\tmpframe{\includegraphics[width=\linewidth]{TODO-image}}
	\caption{Paralelizmus na úrovni systémů. \todo{Vytvořit}}
\end{figure}
\blind[1]
%https://software.intel.com/en-us/articles/designing-the-framework-of-a-parallel-game-engine

\chapter{Návrh komponentního systému}
\todo{Postup seznámení s návrhem top -> down}
\begin{figure}[H]
	\tmpframe{\includegraphics[width=\linewidth]{TODO-image}}
	\caption{Návrhový diagram komponentního systému. \todo{Vytvořit}}
\end{figure}
\blind[1]

\section{Reprezentace entit}
\todo{Entita, id, generace. Alternativy a důvody.}
\blind[3]

\section{Komponenty a jejich nosiče}
\todo{Komponenta POD. Trivial constructor, copy. Paměť a typy nosičů. Typy komponent - tag, normal.}
\blind[4]

\section{Systémy a skupiny}
\todo{Volnost použití systémů. Uživatelem definované chování. Filtry - require, reject. Cache pomocí skupin entit.}
\blind[5]

\section{Paralelní přístup}
\todo{Podporované způsoby paralelizace, návrh, zdůvodnění. Entity paralellization. System parallelization. Diskuze na téma, zda je paralelní přístup potřeba.}
\blind[5]

\section{Přehled komunikace}
\todo{High-level design, propojení. Universe, action-cache.}
\begin{figure}[H]
	\tmpframe{\includegraphics[width=\linewidth]{TODO-image}}
	\caption{Přehled návrhu komponentního systému a komunikace mezi jeho částmi. \todo{Vytvořit - HLB diagram}}
\end{figure}
\blind[3]

\section{Tok řízení}
\todo{Popsat předpokládaný tok řízení, jak se bude používat. Registrace. Iterace. Refresh.}
\begin{figure}[H]
	\tmpframe{\includegraphics[width=\linewidth]{TODO-image}}
	\caption{Digram toku řízení komponentního systému. \todo{Vytvořit - Control flow diagram}}
\end{figure}
\blind[3]


\chapter{Implementace}
\todo{Shrnutí návrhu, zdůvodnění výběru implementačního jazyku C++}
\blind[2]

\section{Použité knihovny a přenositelnost}
\todo{Zdůvodnění nepoužití knihoven. C++14, Linux, Windows.}
\blind[2]

\section{Správa entit}
\todo{Potřebné informace, implementace, datové struktury, optimalizace}
\blind[3]

\section{Komponenty}
\todo{Rekapitulace vlastností, které komponenta musi splnit. Registrace, generování identifikátorů. Uživatelské nosiče}
\blind[3]

\section{Systémy a skupiny}
\todo{Super třída, filtry, přístup - iterace. Přidání systému, volitelná skupina - added/removed.}
\blind[3]

\section{Obnovovací fáze}
\todo{Pořadí refresh. Obnovení aktuality skupin.}
\blind[2]

\section{Podpora paralelizmu}
\todo{2 způsoby paralelizmu - dělení skupin, change-sety}
\blind[3]


\chapter{Vyhodnocení}
\todo{Shrnutí, popis co bude v této kapitole}
\blind[1]

\section{Testování knihovny}
\todo{Unit testing, benchmarky}
\blind[2]

\section{Testované sestavy}
\todo{Hardware sestavy - PC, notebook. Tabulka identifikace}
\blind[1]

\section{Výkonnostní testy}
\todo{Výkon při různých zatíženích - procenta přidaných/odebraných. Grafy.}
\blind[1]
\todo{Porovnání nosičů komponent. Grafy.}
\blind[1]
\todo{Seriové vs paralelní - počet vláken. Grafy.}
\blind[1]

\section{Porovnání}
\todo{Výkonostní porovnání oproti jiným ECS knihovnám, Unity? Grafy.}
\blind[2]

\section{Demo hra}
\todo{Důvod, typ hry}
\blind[1]

\subsection{Návrh}
\todo{Popis principů hry}
\blind[1]
\todo{Komponentní návrh}
\blind[1]

\subsection{Implementace}
\todo{Použité knihovny, základní popis implementace}
\blind[2]

\subsection{Výsledek}
\todo{Poznatky a výsledky získané ze hry. Grafy?}
\blind[2]


\chapter{Závěr}
\todo{Využití komponentních systémů a aktuální stav} 
\blind[1]
\todo{Návrh, prototypy, priority} 
\blind[1]
\todo{Implementace, rozšiřitelnost, přenositelnost}
\blind[1]
\todo{Experimenty, srovnání, demo hra}
\blind[1]
\todo{Možnosti rozšíření}
\blind[1]

%\begin{figure}
%	\tmpframe{\includegraphics[width=\linewidth]{TODO-image}}
%	\caption{Obrazek \todo{Obrazek}}
%\end{figure}

%=========================================================================
